\documentclass[a4paper]{article} 
\usepackage{bold-extra}

\usepackage{comment}
\usepackage{multicol}
\usepackage{graphicx}
\usepackage{amsmath}
\usepackage{amsfonts}
\usepackage[utf8]{inputenc}
\usepackage[portuguese]{babel}  
\usepackage[tmargin=2cm,bmargin=1.8cm, hmargin=0.8in]{geometry}
\usepackage{indentfirst}
\usepackage{abstract}
\usepackage{sectsty}
\usepackage{booktabs}
\usepackage{floatrow}
\usepackage{tabularx}
\usepackage{changepage}
\usepackage{multirow}
\usepackage{microtype}
\usepackage{vwcol}
\usepackage{enumerate} 
\usepackage{esvect} 
\usepackage{siunitx}
\usepackage{caption}
\usepackage{hyperref}
\usepackage{enumitem}
\usepackage{subfigure}
\usepackage{bold-extra}
\usepackage{titlesec}
\usepackage{secdot}
\usepackage{verbatim}
\usepackage{commath}
\usepackage[T1]{fontenc}
\floatsetup[table]{capposition=top}
\sectionfont{\large\sc\selectfont}
\subsectionfont{\hspace{0.2mm}\fontsize{10}{10}\sc\selectfont}
\long\def\/*#1*/{}
\setlength{\skip\footins}{1cm}
\newcommand*{\sectionformat}{\centering}

\setcounter{secnumdepth}{0}
% ESTE FICHEIRO FOI ESCRITO NO SITE SHARELATEX.COM
\begin{document}
\renewcommand\thesubsection{\Alpha.{subsection}}

\title{\bf\LARGE{\vspace{-19mm} Manual do Projecto Final}\vspace{-1.5mm}}
\author{{ {Ricardo Amadeu (83853)} {Tomé Santos (90429)}} \vspace{1mm}\\\textit{\small{Instituto Superior Técnico}}\vspace{-1mm}\\ \textit{\small{Mestrado Integrado em Engenharia Física Tecnológica}}\vspace{-1mm}
\\ \textit{\small{Programação}}\vspace{-1mm}
\date{\normalsize{1º Semestre 2017/2018}}}

\maketitle

\vspace{-6mm}
\renewcommand{\abstractname}{Resumo}
\begin{abstract} \vspace{-1mm}

\par \bf Este documento apresenta uma descrição breve do projecto realizado no âmbito da Unidade Curricular de Programação no Instituto Superior Técnico, desenvolvido em linguagem de programação C em ambiente de janelas com recurso à biblioteca GTK+. Este manual tem como principal objectivo a explicação das funcionalidades e potencialidades do programa, não sendo explicado exaustivamente o código do mesmo. O projecto consiste num programa que permite ao utilizador a visualização de uma simulação do movimento dum sistema de duas molas e um pêndulo. Neste texto encontram-se referidas as componentes constituintes da GUI (\textit{Graphical User Interface}) que permitem a interação do utilizador com o programa, bem como as suas principais funcionalidades.

\end{abstract}

\vspace{2mm}
\section{\bf A. Introdução}
\par O programa desenvolvido possui várias componentes, sendo que umas permitem ao utilizador a introdução de dados ou a manipulação das tarefas a executar pelo programa (\textit{input}), enquanto que outras têm como objectivo a visualização da informação processada pelo programa (\textit{output}). A janela apresentada ao utilizador (fig.1) apresenta-se organizada em três secções: 
\begin{itemize}
\item Gráficos: responsável pela escolha e apresentação de gráficos em tempo real que mostram as variáveis do movimento em função do tempo ou a posição de uma das massas em função do ângulo que o pêndulo faz com a vertical;
\item Simulação do Movimento: possibilita a visualização gráfica do movimento em questão e apresenta funcionalidades que permitem o controlo da velocidade da simulação e de condições fronteira do sistema;
\item Grandezas Físicas e Condições Iniciais: painel de controlo que permite a alteração das grandezas físicas e das condições iniciais do sistema, apresentando também botões que permitem iniciar/parar a simulação, aplicar as alterações nas condições iniciais e de reset.
\end{itemize}
\vspace{2mm}

\begin{figure}[H]
\caption{Janela do programa}
\begin{center}
\includegraphics[scale = 0.4]{janela.png}
\end{center}
\end{figure}
\clearpage 
\section{\bf B. Secções e funcionalidades}
\par Nos subcapítulos seguintes encontra-se uma explicação mais elaborada sobre cada uma das secções referidas na secção anterior. Os componentes referidos de seguida foram dispostos em várias \textit{boxes} e \textit{frames} de modo a melhorar o aspecto e organização da janela.
\subsection{ B.1. Gráficos}
\par Esta secção é constituída por duas partes iguais, sendo cada uma constituída por uma \textit{combo box}, que permite a visualização do gráfico pretendido, um \textit{spin button} que permite alterar o valor da menor divisão da escala, um \textit{button} que permite aplicar o valor do \textit{spin button} na escala do gráfico e uma \textit{drawing area} que desenha o gráfico pretendido. A \textit{combo box} altera os valores atual e limites permitidos pelo \textit{spin button} do respectivo gráfico. As escalas referentes ao ângulo de desvio do pêndulo com a vertical são fixos, uma vez que $\theta$ pertence ao intervalo $[ -\pi , +\pi ]$, e as unidades das escalas são alteradas automaticamente. Nos gráficos que apresentam as variáveis do movimento em função do tempo é apresentada no eixo horizontal uma escala de tempo, desde que o utilizador não altere a velocidade da simulação.

\subsection{ B.2. Simulação do Movimento}
\par Esta secção é constituída por uma  \textit{drawing area} que apresenta a simulação do movimento. Para além disto, contém dois \textit{buttons} que aceleram ou abrandam a simulação, sendo os valores possíveis para a velocidade desta $\times 0.01, \times0.1, \times1$ ou $\times10$ da velocidade em tempo real. Possui também um \textit{toggle button} que controla a sensibilidade dos botões anteriormente referidos. Ao iniciar a simulação, o valor deste botão por \textit{default} é \textbf{disabled}, de modo a permitir a visualização da escala do tempo nos gráficos. Caso se ative o respectivo botão, esta funcionalidade é bloqueada de modo a impedir a apresentação de dados incorretos ao utilizador (programa não se encontra preparado para lidar com esta situação). A informação respeitante à velocidade de simulação é apresentada num \textit{label}. 
\par Esta secção apresenta uma funcionalidade extra, um \textit{toggle button} que permite ativar/desativar a colisão da massa retangular com os extremos da mola. Esta funcionalidade é susceptível a erros de cálculo internos ao programa, pelo que tem um mecanismo de segurança. Este mecanismo, caso o botão esteja ativado, desativa-o se a massa retangular ultrapassar determinadas posições extremas, e impede que o botão seja ativado nestas mesmas condições extraordinárias.

\subsection{ B.3. Grandezas Físicas e Condições Iniciais}
\par Esta secção é constituída por duas \textit{frames}, sendo estas tituladas "Grandezas Físicas" e "Condições Iniciais", cada uma contendo \textit{spin buttons} que permitem a alteração das grandezas físicas e das condições iniciais do sistema. As alterações das grandezas físicas são realizadas em tempo real, sendo estas visíveis na \textit{drawing area} que apresenta a simulação do movimento, enquanto que as das condições iniciais são induzidas em tempo real apenas durante a pausa da simulação ou premindo no \textit{button} "Aplicar", que tem o \textit{shortcut} "ALT-P". Para voltar ao estado inicial do programa, basta clicar no \textit{button} "Reset".
\par O \textit{button} "Iniciar/Pausar" tem algumas funcionalidades: ao parar, os gráficos ficam parados, permite parar a simulação em qualquer instante, mesmo com condições do movimento extremas e permite alterar a velocidade da simulação; ao iniciar, os gráficos reiniciam, a informação dos \textit{spin buttons} é registada internamente e desativa o \textit{toggle button} que permite alterar a velocidade da simulação.

\section{\bf  Referências}

  \bibitem{notes} [1] Samuel Eleutério (2015). {\em Algumas Notas Básicas sobre \LaTeX}
  
\end{document}
